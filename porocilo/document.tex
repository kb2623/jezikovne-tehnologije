\documentclass{acm_proc_article-sp}
\usepackage[utf8]{inputenc}
\usepackage[slovene]{babel}
\usepackage{hyperref}

\begin{document}

\title{Gradnja obširne knjižnice algoritmov po vzorih iz narave}
\subtitle{[Projekt pri predmetu Povezljivi sistemi in inteligentne storitve]}

\numberofauthors{1}

\author{
\alignauthor
Klemen Berkovič\\
       \affaddr{Fakulteta za računalništvo in informatiko}\\
       \affaddr{Maribor, Slovenija}\\
       \email{klemen.berkovic@student.um.si}
\alignauthor
}

\maketitle

\begin{abstract}
Algoritmi po vzorih iz narave so metode, za reševanje težkih optimizacijskih problemov.
Delimo jih v evolucijske algoritme (angl. Evolutionary Algorithms, krajše EA) in algoritme inteligence rojev (angl. Swarm Intelligence, krajše SI).
Medtem ko so evolucijski algoritmi že preživeli svojo dobo zorenja in imajo dobro izdelano klasifikacijo z jasno definiranimi vrstami algoritmov (tj. genetski algoritmi, genetsko programiranje, evolucijske strategije in evolucijsko programiranje), pa razvoj na področju intelidence rojev še zdaleč ni zaključen.
Do danes je bilo tako razvitih prek 200 različnih algoritmov te vrste.
Ker do sedaj še ne obstaja neke robustne knjižnice, ki bi vključevala velik nabor teh algoritmov, je ta projekt namenjen ravno temu.
\end{abstract}

\keywords{inteligence rojev, evolucijski algoritmi, genetski algoritmi, genetsko programiranje, evolucijske strategije, evolucijsko programiranje}

\section{Uvod}
Ker je projekt obširen, priporočamo, da na njem dela čim več študentov.
Pričakovani cilji so dobro dokumentirana knjižnica, ki bo vsebovala čim več algoritmov po vzorih iz narave.
Priporočljivo je, da študenti kodirajo v enem izmed naslednjih programskih jezikov: Python/Ruby/C++.
Rezultate tega projekta lahko javno objavimo na Github-u, da ga lahko uporabljajo in nadgrajujejo tudi drugi uporabniki.

Seznam algoritmov~\cite{DBLP:journals/corr/FisterYFBF13}.

\section{Opis rešitve}
V svojem projektu bom poizkušal implementirati evolucijska algoritma objavljena v člankih~\cite{Brest:2006:SCP:2221387.2221799, Drias:2005:CBS:2152530.2152575}, ter algoritma inteligence rojev objavljena v člankih~\cite{Karaboga2007, Yang2010}.
Za implementacijo rešitve bom uporabil programski jezik Python.
Projekt bom objavil na spletnem mestu GitHub na naslovu \url{https://github.com/kb2623/agents}.

\bibliographystyle{abbrv}
\bibliography{sigproc} 

\end{document}}